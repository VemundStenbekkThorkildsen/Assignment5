\documentclass[10pt,a4paper]{article}
\usepackage[utf8]{inputenc}

\usepackage{amsmath}
\usepackage{amsfonts}
\usepackage{amssymb}
\usepackage{graphicx}
\usepackage{listings}
\usepackage{refstyle}
\usepackage{wasysym}


\lstset{numbers=left,
	title=\lstname,
	numberstyle=\tiny, 
	breaklines=true,
	tabsize=4,
	language=Python,
	morekeywords={with,super,as},,
	frame=single,
	basicstyle=\footnotesize\tt,
	commentstyle=\color{comment},
	keywordstyle=\color{keyword},
	stringstyle=\color{string},
	backgroundcolor=\color{white},
	showstringspaces=false,
	numbers=left,
	numbersep=5pt,
	literate=
		{æ}{{\ae}}1
		{å}{{\aa}}1
		{ø}{{\o}}1
		{Æ}{{\AE}}1
		{Å}{{\AA}}1
		{Ø}{{\O}}1
	}

\usepackage{bm}
\usepackage{hyperref}
\usepackage[margin=1.25 in]{geometry}
\usepackage[usenames, dvipsnames]{color}
\usepackage{float}
\usepackage{commath}

\begin{document}
\begin{center}

{\LARGE\bf
FYS4150\\
\vspace{0.5cm}
Project 5, deadline December 10.
}
 \includegraphics[scale=0.075]{uio.png}\\
Author: Robin David Kifle, Sander Wågønes Losnedahl, Sigmund Slang, Vemund Stenbekk Thorkildsen\\
\vspace{1cm}
{\LARGE\bf
Abstract
}\\
\end{center}


\newpage
{\LARGE\bf
Introduction
}\\



\newpage
{\LARGE\bf
Method
}\\

\noindent The implicit schemes are differential equations that can be rewritten as a set of linear equations. The Euler forward, Euler backward and Crank-Nicolson scheme are all implicit and can therefore be written in term of linear equations after scaling:
\\
Euler backwards, first derivative:
\\
$$
u_t \approx \frac{u(x_i,t_j) - u(x_i,t_j - \Delta t)}{\Delta t}
$$

\noindent Second derivative:

$$
u_{xx} \approx \frac{u(x_i + \Delta x,t_j) - 2u(x_i,t_j) + u(x_i - \Delta x,t_j)}{\Delta x^2}
$$

\noindent One can scale the above equation by $\alpha = \Delta t / \Delta x^2$, so the equation only depends on one scaled variable such that:

$$
u_{i,j-1} = -\alpha u_{i-1,j} + (1 + 2\alpha)u_{i,j} - \alpha u_{i+1,j}
$$

\noindent Now the differential equation can be written as a set of linear equations with a matrix $A$ times a vector $V_j$ such that: $AV_j = V_{j-1}$, where $A$ defined from the above differential equations:

$$
A = \begin{bmatrix}
1 + 2\alpha & -\alpha & 0 & 0 &\cdots\\
-\alpha & 1 + 2\alpha & -\alpha & 0 & \cdots\\
\cdots & \cdots & \cdots & \cdots & \cdots\\
0 & 0 & \cdots & -\alpha & 1 + 2\alpha\\

\end{bmatrix}
$$

\noindent It is now possible to find the previous vector $V_{j-1}$ when we already know what $V_j$, which we already know due to our initial conditions. A more generalized equation can be written as:

$$
A^{-1}(AV_j) = A^{-1}(V_{j-1})
$$
\noindent and if we keep multiplying by $A^{-1}$ we get the implicit scheme:
$$
V_j = A^{-j}V_0
$$

\noindent A very similar process can be applied to the Euler forward method:

$$
u_t = \frac{u(x_i,t_j + \Delta t) - u(x_i,t_j)}{\Delta t}
$$

$$
u_{xx} = 
$$


\newpage
{\LARGE\bf
Results
}\\








\newpage
{\LARGE\bf
Discussion
}\\







\newpage
{\LARGE\bf
Concluding remarks
}\\





\newpage
{\LARGE\bf
Reference list
}\\


\end{document}