\documentclass[10pt,a4paper]{article}
\usepackage[utf8]{inputenc}

\usepackage{amsmath}
\usepackage{amsfonts}
\usepackage{amssymb}
\usepackage{graphicx}
\usepackage{listings}
\usepackage{refstyle}
\usepackage{wasysym}


\lstset{numbers=left,
	title=\lstname,
	numberstyle=\tiny, 
	breaklines=true,
	tabsize=4,
	language=Python,
	morekeywords={with,super,as},,
	frame=single,
	basicstyle=\footnotesize\tt,
	commentstyle=\color{comment},
	keywordstyle=\color{keyword},
	stringstyle=\color{string},
	backgroundcolor=\color{white},
	showstringspaces=false,
	numbers=left,
	numbersep=5pt,
	literate=
		{æ}{{\ae}}1
		{å}{{\aa}}1
		{ø}{{\o}}1
		{Æ}{{\AE}}1
		{Å}{{\AA}}1
		{Ø}{{\O}}1
	}

\usepackage{bm}
\usepackage{hyperref}
\usepackage[margin=1.25 in]{geometry}
\usepackage[usenames, dvipsnames]{color}
\usepackage{float}
\usepackage{commath}

\begin{document}
\begin{center}

{\LARGE\bf
FYS4150\\
\vspace{0.5cm}
Project 5, deadline December 10.
}
 \includegraphics[scale=0.075]{figures/uio.png}\\
Author: Robin David Kifle, Sander Wågønes Losnedahl, Sigmund Slang, Vemund Stenbekk Thorkildsen\\
\vspace{1cm}
{\LARGE\bf
Abstract
}\\
\end{center}


\newpage
{\LARGE\bf
Introduction
}\\



\newpage
{\LARGE\bf
Method
}\\

\noindent The implicit schemes are differential equations that can be rewritten as a set of linear equations. The Euler forward is explicit, the Euler backward is implicit, while the Crank-Nicolson scheme is a combination of the two preceding schemes. These systems can be rewritten as sets of linear equations. \\


\noindent The Euler backwards scheme is implicit, as it uses the current step $i$, and a later step $i+1$ to derive the previous step $i-1$. 
\\
\begin{equation}
u_t \approx \frac{u(x_i,t_j) - u(x_i,t_j - \Delta t)}{\Delta t}
\end{equation}

\begin{equation}
u_{xx} \approx \frac{u(x_i + \Delta x,t_j) - 2u(x_i,t_j) + u(x_i - \Delta x,t_j)}{\Delta x^2}
\end{equation}

\noindent It is possible to scale the above equation by $\alpha = \Delta t / \Delta x^2$, so the equation only depends on one scaled variable. This leads to:

\begin{equation}
u_{i,j-1} = -\alpha u_{i-1,j} + (1 + 2\alpha)u_{i,j} - \alpha u_{i+1,j}
\end{equation}

\noindent Now the differential equation can be written as a set of linear equations with a matrix $A$ times a vector $V_j$ such that $AV_j = V_{j-1}$. Where $A$, defined from the above differential equations take the form:

\begin{equation}
A = \begin{bmatrix}
1 + 2\alpha & -\alpha & 0 & 0 &\cdots\\
-\alpha & 1 + 2\alpha & -\alpha & 0 & \cdots\\
\cdots & \cdots & \cdots & \cdots & \cdots\\
0 & 0 & \cdots & -\alpha & 1 + 2\alpha\\

\end{bmatrix}
\end{equation}

\noindent It is now possible to find the previous vector $V_{j-1}$ when we already know what $V_j$, which we already know due to our initial conditions. A more generalized equation can be written as:

\begin{equation}
A^{-1}(AV_j) = A^{-1}(V_{j-1})
\end{equation}
\noindent and if we keep multiplying by $A^{-1}$ we get the implicit scheme:

\begin{equation}
V_j = A^{-j}V_0
\end{equation}

\noindent A very similar process can be applied to the Euler forward method, but this scheme is explicit:

\begin{equation}
u_t = \frac{u(x_i,t_j + \Delta t) - u(x_i,t_j)}{\Delta t}
\end{equation}

\begin{equation}
u_{xx} \approx \frac{u(x_i + \Delta x,t_j) - 2u(x_i,t_j) + u(x_i - \Delta x,t_j)}{\Delta x^2} 
\end{equation}

\begin{equation}
u_{i,j-1} = \alpha u_{i-1,j} + (1 - 2\alpha)u_{i,j} + \alpha u_{i+1,j}
\end{equation}

\begin{equation}
A = \begin{bmatrix}
1 - 2\alpha & \alpha & 0 & 0 &\cdots\\
\alpha & 1 - 2\alpha & \alpha & 0 & \cdots\\
\cdots & \cdots & \cdots & \cdots & \cdots\\
0 & 0 & \cdots & \alpha & 1 - 2\alpha\\

\end{bmatrix}
\end{equation}

\noindent such that:

\begin{equation}
A^{-1}(AV_j) = A^{-1}(V_{j-1})
\end{equation}

\noindent We generalize again and get:

\begin{equation}
V_j = A^{-j}V_0
\end{equation}

\noindent The Crank-Nicolson scheme is is a combination of both implicit and explicit schemes, namely the Euler forward and Euler backward method.

\begin{equation}
\frac{\theta}{\Delta x^2}(u_{i-1,j} - 2u_{i,j} + u_{i+1,j}) + \frac{1 - \theta}{\Delta x^2}(u_{i+1,j-1} - 2u_{i,j-1} + u_{i-1,j-1}) = \frac{1}{\Delta t}(u_{i,j} - u_{i,j-1})
\end{equation}

\noindent where $\theta$ determines whether the scheme is explicit when $\theta = 0$, or implicit when $\theta = 1$. However, it is when $\theta = 1/2$ where we have the actual Crank-Nicolson scheme which is stable for all $\Delta x$ and $\Delta t$. To derive the Crank-Nicolson scheme we begin with the forward Euler method and Taylor expand $u(x,t + \delta t)$, $u(x + \delta x,t)$, $u(x - \delta x,t)$, $u(x + \delta x, t + \Delta t)$ and $u(x + \delta x, t + \Delta t)$ for $t + \Delta t/2$.\\

\noindent Again we scale the equation with $\alpha = \frac{\Delta t}{\Delta x^2}$ which results in the following equation:

\begin{equation}
-\alpha u_{i-1,j} + (2 + 2\alpha)u_{i,j} -\alpha u_{i+1,j} = \alpha u_{i-1,j-1} + (2-2\alpha)u_{i,j-1} + \alpha u_{i+1,j-1}
\end{equation}

\noindent which can be rewritten as

\begin{equation}
(2I + \alpha B)V_j = (2I - \alpha B)V_{j-1}
\end{equation}

\begin{equation}
V_j = (2I + \alpha B)^{-1}(2I - \alpha B)V_{j-1}
\end{equation}

\noindent where I is the identity matrix and B is given by:

\begin{equation}
B = \begin{bmatrix}
2 & -1 & 0 & \cdots &0\\
-1 & 2 & -1 & 0 & \vdots\\
\vdots & \ddots & \ddots & \ddots & \vdots\\
\vdots & 0 & \ddots & \ddots & -1\\
0 & \cdots & \cdots & -1 & 2\\
\end{bmatrix}
\end{equation}






\newpage
{\LARGE\bf
Results
}\\


\noindent The truncation errors and stability is calculated in the Taylor expansion and these values are shown in the table below:


\begin{table}[H]
\centering
\begin{tabular}{|c|c|c|}
\hline
Method & Truncation error & Stability for\\
\hline
Euler Forward & $\Delta x^2$, $\Delta t$ & $\Delta x^2$ and $\Delta t^2$\\
\hline
Euler Backward & $\Delta x^2$, $\Delta t$ & $\Delta x^2$ and $\Delta t^2$\\
\hline
Crank-Nicolson & $\Delta x^2$, $\Delta t^2$ & $\frac{1}{2} \Delta x^2 \geq \Delta t$\\
\hline
\end{tabular}
\caption{Truncation errors and stability for the three methods}
\label{truncstab}
\end{table}





\newpage
{\LARGE\bf
Discussion
}\\

From Table \ref{truncstab}





\newpage
{\LARGE\bf
Concluding remarks
}\\





\newpage
{\LARGE\bf
Reference list
}\\


\end{document}